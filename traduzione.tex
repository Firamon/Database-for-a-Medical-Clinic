\documentclass[11pt]{article}
\usepackage{listings}
\usepackage{xcolor}
\usepackage{graphicx}
\usepackage{float}

\begin{document}
\title{Matematica in Latex}
\author{Calabrigo Massimo}
\date{\today}
\maketitle

\tableofcontents

\section{Relazionale}
Una volta ristrutturato, l'ER è pronto per essere tradotto in relazionale, e quindi le entità e le relazioni dell'ER, fanno posto alle relazioni del relazionale, e subito dopo abbiamo 
validato la nuova struttura, verificando che il relazionale appena prodotto rispettasse le forme normali.

\subsection{Traduzione}

Per prima cosa abbiamo iniziato identificando le entità principali dell'ER che si sarebbero trasformate in singole relazioni, e che avevamo già identificato nella fase di ristrutturazione come medico, membroPersonaleAusiliario, paziente, terapiaProlungata, ... .

Poi ci siamo concentrati sulle relazioni molti a molti (dell'ER), per trasformarle in relazioni (in relazionale), anche queste erano già state identificate nella fase precedente di ristrutturazione; relazioni come: AusiliarioSeduta, MedicoRegistra, Specializzare, ... .

Una volta ottenute le relazioni (del relazionale), abbiamo assegnato i vincoli di chiave primaria, secondo i campi di chiave primaria dell'ER, e le chiavi esterne, facendo attenzione a porle nelle relazioni (del relazionale) ottenute da relazioni molti a molti (dell'ER), per esempio:
\begin{itemize}
    \item la relazione MedicoRegistra (molti a molti), è diventata: "MedicoRegistra(mese (ext), codice medico (ext), ordinario, straordinario)";
    \item la relazione Specializzare (molti a molti), è diventata: "Specializzare(CodiceMedico(ext), tipoDiSpecializzazione (ext))".
    \item \dots
\end{itemize}

Il lettore può andare a vedere il documento "Relazionale.docx", per trovare tutte le relazioni tradotte.

\subsection{Validazione e forme normali}

\textbf{prima forma normale}: Lo schema relazionale rispetta la prima forma normale sono stati eliminati in fase di ristrutturazione gli attributi multivalore, 
e di conseguenza non esistono campi di riga i e colonna j di nessuna delle istanze delle relazioni dello schema che abbiano più di un valore.\\

\textbf{Seconda forma normale}: Tutti gli attributi non chiave dipendono (direttamente o tramite catene di dipendenze), dalla chiave primaria completa per ogni relazione, qui scriviamo le più articolate:

\begin{itemize}
    \item Medico: tariffa oraria e percentuale incassi non dipendono da tipo perché per uno stesso tipo di medico, possono esserci diverse percentuali o tariffe orarie, mentre per uno stesso codice medico, ci sono le stesse percentuali incassi e tariffa oraria.
    \begin{itemize}
        \item codice medico $\rightarrow$ recapito telefonico
        \item codice medico $\rightarrow$ indirizzo
        \item codice medico $\rightarrow$ nome
        \item codice medico $\rightarrow$ cognome
        \item codice medico $\rightarrow$ tipo
        \item codice medico $\rightarrow$ percentuale incassi
        \item codice medico $\rightarrow$ tariffa oraria
    \end{itemize}

    \item Seduta:
    \begin{itemize}
        \item Cf $\rightarrow$ data
        \item Cf $\rightarrow$ ora
        \item Cf $\rightarrow$ ambulatorio
    \end{itemize}
    \item MembroPersonaleAusiliario:
    \begin{itemize}
        \item Codice medico $\rightarrow$ nome
        \item Codice medico $\rightarrow$ cognome
        \item Codice medico $\rightarrow$ indirizzo
        \item Codice medico $\rightarrow$ recapito telefonico
        \item Codice medico $\rightarrow$ tipo di personale
    \end{itemize}

    \item AusiliarioRegistra:
    \begin{itemize}
        \item Mese, codice personale $\rightarrow$ ordinario
        \item Mese, codice personale $\rightarrow$ straordinario
    \end{itemize}

    \item Paziente:
    \begin{itemize}
        \item Cf $\rightarrow$ nome
        \item Cf $\rightarrow$ cognome
        \item Cf $\rightarrow$ indirizzo
        \item Cf $\rightarrow$ recapito telefonico
        \item Cf $\rightarrow$ data di nascita
        \item Cf $\rightarrow$ tipo
        \item Cf $\rightarrow$ età
    \end{itemize}

    \item TerapiaProlungata:
    \begin{itemize}
        \item Data di Inizio, cf $\rightarrow$ data di fine
        \item Data di Inizio, cf $\rightarrow$ tipo di terapia
        \item Data di Inizio, cf $\rightarrow$ tipo di specializzazione
        \item Data di Inizio, cf $\rightarrow$ numero appuntamenti
        \item Data di Inizio, cf $\rightarrow$ sconto
    \end{itemize}
\end{itemize}

In realtà è facile vedere che non ci sono catene di dipendenze, quindi le relazioni saranno anche in terza forma normale.\\

\textbf{Terza forma normale}: Dall’analisi effettuata è risultato che ogni attributo non chiave dipende direttamente dalla chiave composta, 
e che non esistono dipendenze transitive, quindi tutte le relazioni analizzate sono in terza forma normale.\\
\\
Visto che il relazionale rispetta le forme normali, lo riteniamo validato e possiamo passare alla fase successiva di creazione degli indici.

\end{document}