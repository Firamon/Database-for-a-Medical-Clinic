\documentclass[11pt]{article}
\usepackage{listings}
\usepackage{xcolor}

\definecolor{codegreen}{rgb}{0,0.6,0}
\definecolor{codegray}{rgb}{0.5,0.5,0.5}
\definecolor{codepurple}{rgb}{0.58,0,0.82}
\definecolor{backcolour}{rgb}{0.95,0.95,0.92}

\lstdefinestyle{mystyle}{
    backgroundcolor=\color{backcolour},   
    commentstyle=\color{codegreen},
    keywordstyle=\color{magenta},
    numberstyle=\tiny\color{codegray},
    stringstyle=\color{codepurple},
    basicstyle=\ttfamily\footnotesize,
    breakatwhitespace=false,         
    breaklines=true,                 
    captionpos=b,                    
    keepspaces=true,                 
    numbers=left,                    
    numbersep=5pt,                  
    showspaces=false,                
    showstringspaces=false,
    showtabs=false,                  
    tabsize=2
}

\lstset{style=mystyle}

\begin{document}
\title{Matematica in Latex}
\author{Calabrigo Massimo}
\date{\today}
\maketitle

\tableofcontents

\section{Requisiti e Specifiche}
\subsection{Workflow e fase di specificazione}
\subsection{Requisiti}
\subsection{Specifiche}


\section{ER e relazionale}
\subsection{ER}
\subsubsection{Stesura}
\subsubsection{Ristrutturazione}
\subsubsection{Analisi delle ridondanze}

\subsection{Relazionale}
\subsubsection{Traduzione}
\subsubsection{Validazione e forme normali}


\section{Progettazione fisica}
\subsection{Scelta degli indici}


\section{Alcuni Trigger e Query}
\subsection{Trigger}
\subsection{query}
\begin{lstlisting}[language=SQL]
    -- tutti i medici che hanno visitato il paziente ABCDEF
    select codiceMedico, nome, cognome
    from medico m
    where codiceMedico = any (select codiceMedico
        from medicoSeduta
        where cf = 'ABCDEF')
    or codiceMedico = any (select codiceMedico
        from medicoAppuntamento
        where cf = 'ABCDEF');
\end{lstlisting}


\section{Popolazione ed analisi}
\subsection{Popolazione}
\subsubsection{Snippets}

\subsection{Analisi e grafici}


\section{Conclusioni}

\end{document}
